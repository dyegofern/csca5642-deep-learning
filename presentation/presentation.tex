\documentclass[aspectratio=169]{beamer}

% Theme and Color Scheme
\usetheme{Madrid}
\usecolortheme{default}

% University of Colorado Boulder Colors
\definecolor{PrimaryColor}{RGB}{207,184,124}
\definecolor{SecondaryColor}{RGB}{0,0,0}
\definecolor{TertiaryColor}{RGB}{86,90,92}
\definecolor{QuaternaryColor}{RGB}{162,164,163}

% Apply color scheme
\setbeamercolor{palette primary}{bg=PrimaryColor,fg=black}
\setbeamercolor{palette secondary}{bg=TertiaryColor,fg=white}
\setbeamercolor{palette tertiary}{bg=SecondaryColor,fg=PrimaryColor}
\setbeamercolor{palette quaternary}{bg=PrimaryColor,fg=black}
\setbeamercolor{structure}{fg=SecondaryColor}
\setbeamercolor{section in toc}{fg=SecondaryColor}
\setbeamercolor{subsection in head/foot}{bg=TertiaryColor,fg=white}

% Title colors
\setbeamercolor{title}{fg=PrimaryColor,bg=SecondaryColor}
\setbeamercolor{frametitle}{fg=black,bg=PrimaryColor}

% Block colors
\setbeamercolor{block title}{bg=PrimaryColor,fg=black}
\setbeamercolor{block body}{bg=QuaternaryColor!20,fg=black}

% Alert block colors
\setbeamercolor{block title alerted}{bg=SecondaryColor,fg=PrimaryColor}
\setbeamercolor{block body alerted}{bg=TertiaryColor!20,fg=black}

% Item colors
\setbeamercolor{item}{fg=PrimaryColor}
\setbeamercolor{subitem}{fg=TertiaryColor}

% Packages
\usepackage[utf8]{inputenc}
\usepackage[T1]{fontenc}
\usepackage{graphicx}
\usepackage{booktabs}
\usepackage{amsmath}
\usepackage{hyperref}
\usepackage{tikz}
\usepackage{pgfplots}
\pgfplotsset{compat=1.17}
\usepackage{listings}
\usepackage{xcolor}

% Configure listings to handle UTF-8
\lstset{
    inputencoding=utf8,
    extendedchars=true,
    literate=
        {²}{{\textsuperscript{2}}}1
        {³}{{\textsuperscript{3}}}1
        {°}{{\textdegree}}1
        {±}{{\textpm}}1
        {×}{{\texttimes}}1
        {÷}{{\textdiv}}1
        {≤}{{$\leq$}}1
        {≥}{{$\geq$}}1
        {≠}{{$\neq$}}1
        {∞}{{$\infty$}}1
        {α}{{$\alpha$}}1
        {β}{{$\beta$}}1
        {γ}{{$\gamma$}}1
        {δ}{{$\delta$}}1
        {ε}{{$\epsilon$}}1
        {λ}{{$\lambda$}}1
        {μ}{{$\mu$}}1
        {σ}{{$\sigma$}}1
        {π}{{$\pi$}}1
        {→}{{$\rightarrow$}}1
        {←}{{$\leftarrow$}}1
}

% Define CODE macro for Python code
\lstdefinestyle{pythonstyle}{
    language=Python,
    basicstyle=\ttfamily\scriptsize,
    keywordstyle=\color{blue}\bfseries,
    commentstyle=\color{green!50!black}\itshape,
    stringstyle=\color{red},
    numbers=left,
    numberstyle=\tiny\color{gray},
    stepnumber=1,
    numbersep=5pt,
    backgroundcolor=\color{white},
    showspaces=false,
    showstringspaces=false,
    showtabs=false,
    frame=single,
    tabsize=4,
    captionpos=b,
    breaklines=true,
    breakatwhitespace=false,
    escapeinside={(*@}{@*)},
    xleftmargin=0.5cm,
    xrightmargin=0.5cm,
    inputencoding=utf8,
    extendedchars=true
}

\lstset{style=pythonstyle}

% CODE macro to include external Python files
\newcommand{\CODE}[1]{\lstinputlisting{assets/code/#1}}

% Title Information
\title{Synthetic Brand Generation V2 - Enhanced with Ensemble Methods}
\subtitle{University of Colorado Boulder - CSCA 5642: Introduction to Deep Learning}
\author{Generation}
\institute{University of Colorado Boulder}
\date{\today}

% Logo on title page
\titlegraphic{\includegraphics[height=1.5cm]{assets/logos/cu_logo.svg}}

% Footer customization
\setbeamertemplate{footline}{
  \leavevmode%
  \hbox{%
  \begin{beamercolorbox}[wd=.333333\paperwidth,ht=2.25ex,dp=1ex,center]{author in head/foot}%
    \usebeamerfont{author in head/foot}\insertshortauthor
  \end{beamercolorbox}%
  \begin{beamercolorbox}[wd=.333333\paperwidth,ht=2.25ex,dp=1ex,center]{title in head/foot}%
    \usebeamerfont{title in head/foot}\insertshorttitle
  \end{beamercolorbox}%
  \begin{beamercolorbox}[wd=.333333\paperwidth,ht=2.25ex,dp=1ex,right]{date in head/foot}%
    \usebeamerfont{date in head/foot}\insertshortdate{}\hspace*{2em}
    \insertframenumber{} / \inserttotalframenumber\hspace*{2ex}
  \end{beamercolorbox}}%
  \vskip0pt%
}

\begin{document}

% Title Slide
\begin{frame}
    \titlepage
\end{frame}

% Table of Contents
\begin{frame}{Outline}
    \tableofcontents
\end{frame}


\section{Synthetic Brand Generation V2 - Enhanced with Ensemble Methods}

\begin{frame}{University of Colorado Boulder - CSCA 5642: Introduction to Deep Learning}
\textbf{Dyego Fernandes de Sousa}\\
\end{frame}


\section{Problem Statement}

\begin{frame}
To showcase and demonstrate the application of \textbf{Generative Deep Learning} techniques for synthesizing realistic tabular data with associated text fields.
This project is a continuation of my previous works on \textbf{Supervised Learning} and \textbf{Unsupervised Learning}, you can find more information in the appendix.
\end{frame}

\begin{frame}{Dataset Overview:}
Detailed ESG (Environmenta Social and Governance) dataset to the brand-level
Observations: 3605
Features: 77
\end{frame}

\begin{frame}{Deliverables}
1. \textbf{Generative Adversarial Networks (GANs)}: Implement and train CTGAN for mixed-type tabular data
2. \textbf{Variational Autoencoders (VAEs)}: Apply TVAE as an alternative generative approach
3. \textbf{Statistical Methods}: Integrate Gaussian Copula for correlation structure preservation
4. \textbf{Large Language Models (LLMs)}: Fine-tune GPT-2 and Flan-T5 for conditional text generation
5. \textbf{Ensemble Methods}: Combine multiple generators with optimized weighting
6. \textbf{Evaluation Metrics}: Apply statistical tests (KS, correlation) to assess synthetic data quality
\end{frame}


\section{ML Approach & Architecture}

\begin{frame}
The following diagram illustrates the complete pipeline for synthetic brand data generation:
\end{frame}

\begin{frame}{Deep Learning Models Used}
| Model | Type | Purpose | Key Features |
|-------|------|---------|-------------|
| \textbf{CTGAN} | Conditional GAN | Tabular synthesis | Mode-specific normalization, conditional vector |
| \textbf{TVAE} | Variational Autoencoder | Distribution learning | KL divergence, latent space regularization |
| \textbf{Gaussian Copula} | Statistical | Correlation preservation | Multivariate dependencies |
| \textbf{GPT-2 Medium} | Transformer LLM | Brand name generation | 355M params, fine-tuned |
| \textbf{Flan-T5 Small} | Encoder-Decoder | Conditional text gen | Instruction-following |
\end{frame}

\begin{frame}
!\href{attachment:mermeid1.png}{mermeid1.png}
\end{frame}

\begin{frame}
!\href{attachment:mermeid2.png}{mermeid2.png}
\end{frame}

\begin{frame}
!\href{attachment:mermeid3.png}{mermeid3.png}
\end{frame}


\section{Phase 1: Setup & Installation}


\begin{frame}[fragile]{Code: Cell 7}
    \begin{figure}
        \CODE{cell_007.py}
        \caption{Code from Cell 7}
    \end{figure}
\end{frame}


\begin{frame}[fragile]{Code: Cell 8}
    \begin{figure}
        \CODE{cell_008.py}
        \caption{Code from Cell 8}
    \end{figure}
\end{frame}


\section{Configuration}


\begin{frame}[fragile]{Code: Cell 10}
    \begin{figure}
        \CODE{cell_010.py}
        \caption{Code from Cell 10}
    \end{figure}
\end{frame}


\section{Phase 0: Hyperparameter Tuning (Optional)}

\begin{frame}
This phase uses Optuna to find optimal hyperparameters for the tabular synthesizers.
Set \texttt{RUN\_HYPERPARAMETER\_TUNING = True} to run optimization, or use previously saved best parameters.
\textbf{Tuned Parameters:}
\begin{itemize}
    \item \textbf{CTGAN/TVAE}: epochs, batch\_size, embedding\_dim, generator/discriminator dimensions
    \item \textbf{Ensemble}: weights for each model
    \item \textbf{Generation}: noise\_level, diversity\_temperature
\end{itemize}
\end{frame}


\begin{frame}[fragile]{Code: Cell 12}
    \begin{figure}
        \CODE{cell_012.py}
        \caption{Code from Cell 12}
    \end{figure}
\end{frame}


\begin{frame}[fragile]{Code: Cell 13}
    \begin{figure}
        \CODE{cell_013.py}
        \caption{Code from Cell 13}
    \end{figure}
\end{frame}


\begin{frame}[fragile]{Code: Cell 14}
    \begin{figure}
        \CODE{cell_014.py}
        \caption{Code from Cell 14}
    \end{figure}
\end{frame}


\section{Phase 1: Data Preparation}


\begin{frame}[fragile]{Code: Cell 16}
    \begin{figure}
        \CODE{cell_016.py}
        \caption{Code from Cell 16}
    \end{figure}
\end{frame}


\begin{frame}[fragile]{Code: Cell 17}
    \begin{figure}
        \CODE{cell_017.py}
        \caption{Code from Cell 17}
    \end{figure}
\end{frame}


\begin{frame}[fragile]{Code: Cell 18}
    \begin{figure}
        \CODE{cell_018.py}
        \caption{Code from Cell 18}
    \end{figure}
\end{frame}


\begin{frame}[fragile]{Code: Cell 19}
    \begin{figure}
        \CODE{cell_019.py}
        \caption{Code from Cell 19}
    \end{figure}
\end{frame}


\begin{frame}[fragile]{Code: Cell 20}
    \begin{figure}
        \CODE{cell_020.py}
        \caption{Code from Cell 20}
    \end{figure}
\end{frame}


\begin{frame}{Analysis Result}
    \begin{figure}
        \centering
        \includegraphics[width=0.85\textwidth,height=0.7\textheight,keepaspectratio]{assets/figures/figure_001.png}
        \caption{Output from Cell 20}
    \end{figure}
\end{frame}


\section{Phase 2: Tabular Ensemble Training}

\begin{frame}
Training CTGAN, TVAE, and Gaussian Copula models
\end{frame}


\begin{frame}[fragile]{Code: Cell 22}
    \begin{figure}
        \CODE{cell_022.py}
        \caption{Code from Cell 22}
    \end{figure}
\end{frame}


\begin{frame}[fragile]{Code: Cell 23}
    \begin{figure}
        \CODE{cell_023.py}
        \caption{Code from Cell 23}
    \end{figure}
\end{frame}


\begin{frame}[fragile]{Code: Cell 24}
    \begin{figure}
        \CODE{cell_024.py}
        \caption{Code from Cell 24}
    \end{figure}
\end{frame}


\begin{frame}[fragile]{Code: Cell 25}
    \begin{figure}
        \CODE{cell_025.py}
        \caption{Code from Cell 25}
    \end{figure}
\end{frame}


\section{Phase 3: LLM Ensemble Training}

\begin{frame}
Training GPT-2 Medium and Flan-T5 for brand name generation
\end{frame}


\begin{frame}[fragile]{Code: Cell 27}
    \begin{figure}
        \CODE{cell_027.py}
        \caption{Code from Cell 27}
    \end{figure}
\end{frame}


\begin{frame}[fragile]{Code: Cell 28}
    \begin{figure}
        \CODE{cell_028.py}
        \caption{Code from Cell 28}
    \end{figure}
\end{frame}


\begin{frame}[fragile]{Code: Cell 29}
    \begin{figure}
        \CODE{cell_029.py}
        \caption{Code from Cell 29}
    \end{figure}
\end{frame}


\begin{frame}[fragile]{Code: Cell 30}
    \begin{figure}
        \CODE{cell_030.py}
        \caption{Code from Cell 30}
    \end{figure}
\end{frame}


\section{Phase 4: Synthetic Data Generation}


\begin{frame}[fragile]{Code: Cell 32}
    \begin{figure}
        \CODE{cell_032.py}
        \caption{Code from Cell 32}
    \end{figure}
\end{frame}


\begin{frame}[fragile]{Code: Cell 33}
    \begin{figure}
        \CODE{cell_033.py}
        \caption{Code from Cell 33}
    \end{figure}
\end{frame}


\begin{frame}[fragile]{Code: Cell 34}
    \begin{figure}
        \CODE{cell_034.py}
        \caption{Code from Cell 34}
    \end{figure}
\end{frame}


\begin{frame}[fragile]{Code: Cell 35}
    \begin{figure}
        \CODE{cell_035.py}
        \caption{Code from Cell 35}
    \end{figure}
\end{frame}


\begin{frame}[fragile]{Code: Cell 36}
    \begin{figure}
        \CODE{cell_036.py}
        \caption{Code from Cell 36}
    \end{figure}
\end{frame}


\begin{frame}[fragile]{Code: Cell 37}
    \begin{figure}
        \CODE{cell_037.py}
        \caption{Code from Cell 37}
    \end{figure}
\end{frame}


\begin{frame}[fragile]{Code: Cell 38}
    \begin{figure}
        \CODE{cell_038.py}
        \caption{Code from Cell 38}
    \end{figure}
\end{frame}


\section{Phase 5: Synthetic Data Quality Evaluation}

\begin{frame}
This phase comprehensively evaluates the quality of generated synthetic data through:
1. \textbf{Statistical Fidelity}: Distribution matching (KS tests), correlation preservation
2. \textbf{Visual Analysis}: Distribution overlays, QQ plots, feature comparisons
3. \textbf{Dimensionality Reduction}: PCA and t-SNE projections
4. \textbf{Summary Metrics}: Quality scorecards and radar charts
\end{frame}


\begin{frame}[fragile]{Code: Cell 40}
    \begin{figure}
        \CODE{cell_040.py}
        \caption{Code from Cell 40}
    \end{figure}
\end{frame}


\begin{frame}[fragile]{Code: Cell 41}
    \begin{figure}
        \CODE{cell_041.py}
        \caption{Code from Cell 41}
    \end{figure}
\end{frame}


\begin{frame}[fragile]{Code: Cell 42}
    \begin{figure}
        \CODE{cell_042.py}
        \caption{Code from Cell 42}
    \end{figure}
\end{frame}


\begin{frame}{Analysis Result}
    \begin{figure}
        \centering
        \includegraphics[width=0.85\textwidth,height=0.7\textheight,keepaspectratio]{assets/figures/figure_002.png}
        \caption{Output from Cell 42}
    \end{figure}
\end{frame}


\begin{frame}[fragile]{Code: Cell 43}
    \begin{figure}
        \CODE{cell_043.py}
        \caption{Code from Cell 43}
    \end{figure}
\end{frame}


\begin{frame}{Analysis Result}
    \begin{figure}
        \centering
        \includegraphics[width=0.85\textwidth,height=0.7\textheight,keepaspectratio]{assets/figures/figure_003.png}
        \caption{Output from Cell 43}
    \end{figure}
\end{frame}


\begin{frame}[fragile]{Code: Cell 44}
    \begin{figure}
        \CODE{cell_044.py}
        \caption{Code from Cell 44}
    \end{figure}
\end{frame}


\begin{frame}{Analysis Result}
    \begin{figure}
        \centering
        \includegraphics[width=0.85\textwidth,height=0.7\textheight,keepaspectratio]{assets/figures/figure_004.png}
        \caption{Output from Cell 44}
    \end{figure}
\end{frame}


\begin{frame}[fragile]{Code: Cell 45}
    \begin{figure}
        \CODE{cell_045.py}
        \caption{Code from Cell 45}
    \end{figure}
\end{frame}


\begin{frame}{Analysis Result}
    \begin{figure}
        \centering
        \includegraphics[width=0.85\textwidth,height=0.7\textheight,keepaspectratio]{assets/figures/figure_005.png}
        \caption{Output from Cell 45}
    \end{figure}
\end{frame}


\begin{frame}[fragile,allowframebreaks]{Code: Cell 46}
    \begin{figure}
        \CODE{cell_046.py}
        \caption{Code from Cell 46}
    \end{figure}
\end{frame}


\begin{frame}{Analysis Result}
    \begin{figure}
        \centering
        \includegraphics[width=0.85\textwidth,height=0.7\textheight,keepaspectratio]{assets/figures/figure_006.png}
        \caption{Output from Cell 46}
    \end{figure}
\end{frame}


\begin{frame}[fragile,allowframebreaks]{Code: Cell 47}
    \begin{figure}
        \CODE{cell_047.py}
        \caption{Code from Cell 47}
    \end{figure}
\end{frame}


\begin{frame}{Analysis Result}
    \begin{figure}
        \centering
        \includegraphics[width=0.85\textwidth,height=0.7\textheight,keepaspectratio]{assets/figures/figure_007.png}
        \caption{Output from Cell 47}
    \end{figure}
\end{frame}


\begin{frame}[fragile,allowframebreaks]{Code: Cell 48}
    \begin{figure}
        \CODE{cell_048.py}
        \caption{Code from Cell 48}
    \end{figure}
\end{frame}


\begin{frame}{Analysis Result}
    \begin{figure}
        \centering
        \includegraphics[width=0.85\textwidth,height=0.7\textheight,keepaspectratio]{assets/figures/figure_008.png}
        \caption{Output from Cell 48}
    \end{figure}
\end{frame}


\begin{frame}[fragile,allowframebreaks]{Code: Cell 49}
    \begin{figure}
        \CODE{cell_049.py}
        \caption{Code from Cell 49}
    \end{figure}
\end{frame}


\begin{frame}{Analysis Result}
    \begin{figure}
        \centering
        \includegraphics[width=0.85\textwidth,height=0.7\textheight,keepaspectratio]{assets/figures/figure_009.png}
        \caption{Output from Cell 49}
    \end{figure}
\end{frame}


\section{Clean Up}


\section{Conclusion}

\begin{frame}{What Worked Well}
1. \textbf{Ensemble Architecture for Tabular Data}
\begin{itemize}
    \item The combination of CTGAN, TVAE, and Gaussian Copula provided complementary strengths
    \item TVAE showed the best individual performance (Mean KS: 0.11, 44\% pass rate) for distribution matching
    \item Gaussian Copula excelled at preserving correlation structure (lowest Correlation MSE: 0.0197)
    \item Dynamic weight optimization improved results by shifting weight toward TVAE (~54\%)
\end{itemize}
2. \textbf{Scalable Data Generation Pipeline}
\begin{itemize}
    \item Successfully generated 500 synthetic brand records with stratified company distribution
    \item The pipeline supports conditional generation based on company characteristics
    \item Model persistence to Google Drive enables iterative experimentation
\end{itemize}
3. \textbf{LLM Brand Name Generation}
\begin{itemize}
    \item 95.4\% success rate for unique brand name generation
    \item GPT-2 Medium and Flan-T5 ensemble provided diverse naming suggestions
    \item Memory-efficient sequential loading enabled running on standard GPU
\end{itemize}
4. \textbf{Correlation Structure Preservation}
\begin{itemize}
    \item Mean absolute correlation difference of 0.165 indicates reasonable preservation
    \item PCA visualizations show synthetic data occupies similar feature space as original
\end{itemize}
\end{frame}

\begin{frame}{What Didn't Work Well}
1. \textbf{Distribution Matching Challenges}
\begin{itemize}
    \item Only 29.2\% of features (7/24) passed the KS test for distribution similarity
    \item Particularly poor performance on: \texttt{market\_cap\_billion\_usd} (KS=0.91), \texttt{employees} (KS=0.65), \texttt{greenwashing\_factors} (KS=0.50-0.62)
    \item Heavy-tailed distributions and sparse features remain difficult for GANs
\end{itemize}
2. \textbf{Clustering Quality Degradation}
\begin{itemize}
    \item Silhouette score dropped from 0.713 (original) to 0.682 (augmented)
    \item Davies-Bouldin score increased from 0.387 to 0.421 (worse)
    \item Synthetic data may be introducing noise rather than enhancing cluster structure
\end{itemize}
3. \textbf{LLM Brand Name Quality Issues}
\begin{itemize}
    \item Some generated names are full sentences (e.g., "Nestle is a manufacturer of processed foods")
    \item Competitor name leakage (e.g., "Procter \& Gamble" generated for Bayer)
    \item Repetitive patterns with company name variations (e.g., "Nestle Foods", "Nestle, Inc")
    \item 4.6\% fallback rate indicates generation failures
\end{itemize}
4. \textbf{Ensemble Aggregation Complexity}
\begin{itemize}
    \item Simple weighted averaging may not be optimal for combining diverse model outputs
    \item The ensemble sometimes performed worse than individual models on certain metrics
\end{itemize}
\end{frame}

\begin{frame}{Future Enhancements}
1. \textbf{Improved Tabular Synthesis}
\begin{itemize}
    \item Implement feature-specific preprocessing (log transforms for heavy-tailed distributions)
    \item Use conditional generation with explicit constraints for bounded features
    \item Explore TabDDPM (diffusion models) as an alternative to GAN-based methods
    \item Add post-processing validation to clip unrealistic values
\end{itemize}
2. \textbf{Enhanced LLM Brand Generation}
\begin{itemize}
    \item Fine-tune with negative examples to prevent competitor name generation
    \item Implement stricter output validation and filtering
    \item Use retrieval-augmented generation (RAG) with brand name databases
    \item Add style conditioning for different brand naming conventions (descriptive, invented, founder-based)
\end{itemize}
3. \textbf{Better Evaluation Metrics}
\begin{itemize}
    \item Implement Machine Learning Efficacy tests (train on synthetic, test on real)
    \item Add privacy metrics (nearest neighbor distance, membership inference)
    \item Use domain-specific validity checks for brand attributes
\end{itemize}
4. \textbf{Hyperparameter Optimization}
\begin{itemize}
    \item Extend Optuna tuning to LLM generation parameters (temperature, top\_p, top\_k)
    \item Implement multi-objective optimization balancing distribution matching and correlation preservation
    \item Add early stopping based on validation metrics during training
\end{itemize}
5. \textbf{Architecture Improvements}
\begin{itemize}
    \item Implement attention-based tabular models (TabTransformer, FT-Transformer)
    \item Use hierarchical generation: company → industry → brand attributes → brand name
    \item Add discriminator-based filtering to reject low-quality synthetic samples
\end{itemize}
\end{frame}


\section{References & Bibliography}

\begin{frame}{Synthetic Data Generation}
1. \textbf{CTGAN (Conditional Tabular GAN)}
\begin{itemize}
    \item Xu, L., Skoularidou, M., Cuesta-Infante, A., \& Veeramachaneni, K. (2019). \textit{Modeling Tabular Data using Conditional GAN}. NeurIPS 2019.
    \item Paper: https://arxiv.org/abs/1907.00503
    \item Implementation: \href{https://github.com/sdv-dev/SDV}{SDV Library}
\end{itemize}
2. \textbf{TVAE (Tabular Variational Autoencoder)}
\begin{itemize}
    \item Xu, L., Skoularidou, M., Cuesta-Infante, A., \& Veeramachaneni, K. (2019). \textit{Modeling Tabular Data using Conditional GAN}. NeurIPS 2019.
    \item Part of the same paper as CTGAN, presenting VAE-based alternative
\end{itemize}
3. \textbf{Gaussian Copula}
\begin{itemize}
    \item Patki, N., Wedge, R., \& Veeramachaneni, K. (2016). \textit{The Synthetic Data Vault}. IEEE DSAA 2016.
    \item Paper: https://dai.lids.mit.edu/wp-content/uploads/2018/03/SDV.pdf
\end{itemize}
4. \textbf{SDV (Synthetic Data Vault) Library}
\begin{itemize}
    \item Documentation: https://docs.sdv.dev/sdv/
    \item GitHub: https://github.com/sdv-dev/SDV
\end{itemize}
\end{frame}

\begin{frame}{Language Models for Text Generation}
5. \textbf{GPT-2}
\begin{itemize}
    \item Radford, A., Wu, J., Child, R., Luan, D., Amodei, D., \& Sutskever, I. (2019). \textit{Language Models are Unsupervised Multitask Learners}. OpenAI.
    \item Paper: https://cdn.openai.com/better-language-models/language\_models\_are\_unsupervised\_multitask\_learners.pdf
\end{itemize}
6. \textbf{Flan-T5}
\begin{itemize}
    \item Chung, H. W., et al. (2022). \textit{Scaling Instruction-Finetuned Language Models}. arXiv preprint.
    \item Paper: https://arxiv.org/abs/2210.11416
\end{itemize}
7. \textbf{Hugging Face Transformers}
\begin{itemize}
    \item Wolf, T., et al. (2020). \textit{Transformers: State-of-the-Art Natural Language Processing}. EMNLP 2020.
    \item Documentation: https://huggingface.co/docs/transformers/
\end{itemize}
\end{frame}

\begin{frame}{Evaluation Metrics}
8. \textbf{Kolmogorov-Smirnov Test}
\begin{itemize}
    \item Massey Jr, F. J. (1951). \textit{The Kolmogorov-Smirnov test for goodness of fit}. Journal of the American Statistical Association, 46(253), 68-78.
\end{itemize}
9. \textbf{Silhouette Score}
\begin{itemize}
    \item Rousseeuw, P. J. (1987). \textit{Silhouettes: a graphical aid to the interpretation and validation of cluster analysis}. Journal of Computational and Applied Mathematics, 20, 53-65.
\end{itemize}
10. \textbf{Davies-Bouldin Index}
\begin{itemize}
    \item Davies, D. L., \& Bouldin, D. W. (1979). \textit{A cluster separation measure}. IEEE Transactions on Pattern Analysis and Machine Intelligence, (2), 224-227.
\end{itemize}
\end{frame}

\begin{frame}{Hyperparameter Optimization}
11. \textbf{Optuna}
\begin{itemize}
    \item Akiba, T., Sano, S., Yanase, T., Ohta, T., \& Koyama, M. (2019). \textit{Optuna: A Next-generation Hyperparameter Optimization Framework}. KDD 2019.
    \item Paper: https://arxiv.org/abs/1907.10902
    \item Documentation: https://optuna.org/
\end{itemize}
\end{frame}

\begin{frame}{Related Work on Tabular Data Synthesis}
12. \textbf{TabDDPM (Diffusion Models for Tabular Data)}
\begin{itemize}
    \item Kotelnikov, A., Baranchuk, D., Rubachev, I., \& Babenko, A. (2023). \textit{TabDDPM: Modelling Tabular Data with Diffusion Models}. ICML 2023.
    \item Paper: https://arxiv.org/abs/2209.15421
\end{itemize}
13. \textbf{CTAB-GAN+}
\begin{itemize}
    \item Zhao, Z., Kunar, A., Birke, R., \& Chen, L. Y. (2022). \textit{CTAB-GAN+: Enhancing Tabular Data Synthesis}. arXiv preprint.
    \item Paper: https://arxiv.org/abs/2204.00401
\end{itemize}
14. \textbf{Synthetic Data Generation Survey}
\begin{itemize}
    \item Jordon, J., Yoon, J., \& van der Schaar, M. (2022). \textit{Synthetic Data - what, why and how?} arXiv preprint.
    \item Paper: https://arxiv.org/abs/2205.03257
\end{itemize}
\end{frame}

\begin{frame}{Software & Tools}
\begin{itemize}
    \item \textbf{Python}: https://www.python.org/
    \item \textbf{PyTorch}: https://pytorch.org/
    \item \textbf{Pandas}: https://pandas.pydata.org/
    \item \textbf{Scikit-learn}: https://scikit-learn.org/
    \item \textbf{Matplotlib}: https://matplotlib.org/
    \item \textbf{Seaborn}: https://seaborn.pydata.org/
    \item \textbf{Google Colab}: https://colab.research.google.com/
\end{itemize}
\end{frame}


\begin{frame}[fragile]{Code: Cell 53}
    \begin{figure}
        \CODE{cell_053.py}
        \caption{Code from Cell 53}
    \end{figure}
\end{frame}


\section{Appendix - Mermeid Code}

\begin{frame}
\texttt{}`mermaid
flowchart TB
    subgraph INPUT["📊 Input Data"]
        A[(Brand Dataset<br/>CSV)] --> B[Data Processor]
    end
    subgraph PREPROCESS["🔧 Preprocessing"]
        B --> C[Clean \& Validate]
        C --> D[Feature Engineering]
        D --> E[Train/Test Split]
    end
    subgraph TABULAR["🎲 Tabular Data Generation"]
        E --> F1[CTGAN<br/>Conditional GAN]
        E --> F2[TVAE<br/>Variational Autoencoder]
        E --> F3[Gaussian Copula<br/>Statistical Model]
        F1 --> G[Ensemble<br/>Weighted Averaging]
        F2 --> G
        F3 --> G
    end
    subgraph TEXT["📝 Text Generation"]
        G --> H1[GPT-2 Medium<br/>Fine-tuned LLM]
        G --> H2[Flan-T5 Small<br/>Instruction-tuned]
        H1 --> I[Text Ensemble<br/>Best Selection]
        H2 --> I
    end
    subgraph OUTPUT["✅ Output"]
        I --> J[Synthetic Brands<br/>with Names]
        J --> K[Quality Evaluation]
    end
    subgraph EVAL["📈 Evaluation Metrics"]
        K --> L1[KS Test]
        K --> L2[Correlation]
        K --> L3[PCA/t-SNE]
        K --> L4[Clustering]
    end
    style INPUT fill:\#e1f5fe
    style PREPROCESS fill:\#fff3e0
    style TABULAR fill:\#f3e5f5
    style TEXT fill:\#e8f5e9
    style OUTPUT fill:\#fce4ec
    style EVAL fill:\#fff8e1
\texttt{}`
\end{frame}

\begin{frame}{Ensemble Strategy}
\texttt{}`mermaid
flowchart LR
    subgraph ENSEMBLE["Ensemble Weighting"]
        direction TB
        W1["CTGAN: 40\%"] --> MIX((Weighted<br/>Average))
        W2["TVAE: 35\%"] --> MIX
        W3["Copula: 25\%"] --> MIX
        MIX --> OUT[Synthetic Data]
    end
    style ENSEMBLE fill:\#f5f5f5
    style MIX fill:\#4caf50,color:\#fff
\texttt{}`
\end{frame}

\begin{frame}{Training Pipeline}
\texttt{}`mermaid
sequenceDiagram
    participant D as Dataset
    participant P as Processor
    participant T as Tabular Models
    participant L as LLM Models
    participant E as Evaluator
    D->>P: Load brand\_information.csv
    P->>P: Clean \& preprocess
    P->>T: Training data
    par Train in Parallel
        T->>T: Train CTGAN (300 epochs)
        T->>T: Train TVAE (300 epochs)
        T->>T: Fit Gaussian Copula
    end
    T->>T: Generate synthetic tabular
    T->>L: Tabular features
    par Generate Names
        L->>L: GPT-2 generation
        L->>L: Flan-T5 generation
    end
    L->>E: Complete synthetic data
    E->>E: Statistical tests
    E->>E: Visualization
\texttt{}`
\end{frame}


\section{Tools & Libraries Used}

\begin{frame}{Core Deep Learning & ML Frameworks}
\begin{itemize}
    \item \textbf{PyTorch} - Deep learning framework powering the neural network components of CTGAN and TVAE synthesizers
    \item \textbf{Scikit-learn} - Machine learning utilities for PCA, t-SNE, clustering (AgglomerativeClustering), and evaluation metrics (silhouette score, Davies-Bouldin index)
\end{itemize}
\end{frame}

\begin{frame}{Synthetic Data Generation}
\begin{itemize}
    \item \textbf{SDV (Synthetic Data Vault)} - Primary library for tabular synthetic data generation, providing:
    \item \textbf{CTGAN} - Conditional Tabular GAN for generating realistic tabular data
    \item \textbf{TVAE} - Tabular Variational Autoencoder for distribution-preserving synthesis
    \item \textbf{Gaussian Copula} - Statistical model capturing feature dependencies
\end{itemize}
\end{frame}

\begin{frame}{Natural Language Generation}
\begin{itemize}
    \item \textbf{Hugging Face Transformers} - State-of-the-art NLP library for text generation using pre-trained language models
    \item \textbf{PEFT (Parameter-Efficient Fine-Tuning)} - Efficient fine-tuning techniques for large language models
    \item \textbf{BitsAndBytes} - 8-bit quantization for memory-efficient model loading
    \item \textbf{Accelerate} - Distributed training and mixed precision utilities
    \item \textbf{SentencePiece} - Tokenization library for handling text preprocessing
\end{itemize}
\end{frame}

\begin{frame}{Hyperparameter Optimization}
\begin{itemize}
    \item \textbf{Optuna} - Automated hyperparameter tuning framework using Bayesian optimization for finding optimal model configurations
\end{itemize}
\end{frame}

\begin{frame}{Data Processing & Analysis}
\begin{itemize}
    \item \textbf{Pandas} - Data manipulation and analysis
    \item \textbf{NumPy} - Numerical computing and array operations
    \item \textbf{SciPy} - Statistical tests (Kolmogorov-Smirnov test) for distribution comparison
\end{itemize}
\end{frame}

\begin{frame}{Visualization}
\begin{itemize}
    \item \textbf{Matplotlib} - Core plotting library for creating figures
    \item \textbf{Seaborn} - Statistical data visualization with enhanced aesthetics
    \item \textbf{Plotly} - Interactive visualizations (if used)
\end{itemize}
\end{frame}

\begin{frame}{Environment & Infrastructure}
\begin{itemize}
    \item \textbf{Google Colab} - Cloud-based Jupyter notebook environment with GPU support
    \item \textbf{Google Drive} - Persistent storage for models and outputs
\end{itemize}
\end{frame}



\end{document}
